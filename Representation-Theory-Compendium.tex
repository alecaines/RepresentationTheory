\documentclass[11pt]{book}
\usepackage{compendium_style}


\title{{\bf The Compendium}}
\author{Alexander Caines}
\begin{document}
\maketitle

\newpage
\epigraph{Sometimes, when people are either too lazy or negligent to help you, you've just
gotta hunker down and help yourself!}{The Misfit}
\newpage

\tableofcontents
\newpage

\addtocontents{toc}{\setcounter{tocdepth}{3}}
\section{An Introduction}
\subsection{Inspiration}
As a child, my favorite part of playing video games was figuring out how to beat the boss on a certain level. I would draw upon my 
knowledge of the different enemies I had faced up until that point and realize that the boss fought in a similar fashion---but with 
some new tweaks. I would then alter my already existing strategy to meet the boss's fighting style---and eventually defeat him.\\

Often times when learning a new subject---especially one whose matters build on the developments of another subject---it can be 
helpful to learn in a constructive fashion. Rather than jump into the more advanced subject off the bat, it might be more effective 
to identify the root of the new subject in the developments of the old. Trivially, this is natural to mathematics (as few subjects 
are independent of each other). However, with respect to pedagogy, it easy to find oneself ``lost in the weeds,'' per se. Personally, if 
dependencies between matters are not elaborated upon, it becomes much harder to learn the current subject and anything that might build on 
it. I am writing this book to help myself identify these dependencies with greater ease. I hope it serves the same effect for you as well.

\subsection{Why Representation Theory?}
Representation Theory, to the novice, is a most peculiar field of study. Its conclusions might come off as trivial and proofs needlessly 
tedious and unnecessarily difficult. However, upon further investigation, one will discover that the study of a group's representations
(or those of any other sort of algebra\footnote{Like a field or a ring\dots}) offers them a scaffold to pry into its structure. As 
eloquently described by the user \textbf{Mathmo123} on the \href{https://math.stackexchange.com/questions/1628464/what-is-representation-theory}{mathematics stack exchange},
``Representation theory allows us to translate our viewpoint [of a group] by viewing (a quotient of) our group as a group of linear
maps on a vector space\dots This allows us to tackle problems in group theory using familiar and powerful tools in linear algebra.''
\subsection{Representation Theory}
So, what exactly is representation theory? To be curt, representation theory is the study of different algebraic 
structures \emph{represented} as vector spaces. Specifically, mathematicians who study representation theory search for 
the appropriate \emph{linear transformations} that preserve the patterns observed in an algebraic structure. Because the 
algebra of matrices is more intuitive to most mathematicians, it is valuable to be able to represent algebras using vector spaces.
This book will explore the different liberties afforded by representations and how they can be used to standardize
and in a way unify the fields of combinatorics, numbertheory, and algebra!\\

\newpage

\section{Groups and Group Homomorphisms}
Let's start small and recapitulate a bit of algebra before we dive into playing around with representations.
The simplest algebraic structure is the group, a set $G$ paired with an operation $\otimes$ (often denoted ($G$, $\otimes$)). 
Recall, for a set to be a group with an operation $\otimes$, it must abide by the following properties: $\forall a,b, c \in G$\\
\begin{defin}
	\begin{center}
		\underline{\textbf{Group Properties}}
		\begin{enumerate}
			\item[1.] (Closure under $\otimes$) $a \otimes b \in G$\\
			\item[2.] (Existence of an identity) $\exists e \in G \ni e \otimes a = a$\\
			\item[3.] (Associativity) $(a \otimes b) \otimes c = a \otimes (b \otimes c)$\\
		\end{enumerate}
	\end{center}
\end{defin}
To construct our first representation, we need a group that can be mapped onto by $G$. Take the general linear group of order $d$, $GL_d(\mathbb{C})$\footnote{
We could substitute any field in for $\mathbb{C}$, but $\mathbb{C}$ is the standard in the field and will be used here.} defined,
$$GL_d(\mathbb{C}) := \Bigg\{M \mid M := \Bigg(\begin{matrix} a_{1,1} & a_{1,2} & \dots & a_{1,d}\\ \vdots & \vdots & \vdots & \ddots\\ a_{d,1} & a_{d, 2},
	& \dots & a_{d,d}\end{matrix}\Bigg) \text{, and $M$ is invertible with } a_{i,j} \in \mathbb{C}, \forall i,j \in [0,n]\Bigg\} $$
Using $GL_d(\mathbb{C})$ is standard in representation theory when introducing the essence of a representation for three reasons:

\begin{center}
	\begin{enumerate}
		\item[1.] Because its elements are \emph{square}, $\forall A, B \in GL_d(\mathbb{C})$, $A \cdot B \in GL_d(\mathbb{C})$. (satisfying the closure condition for 
			groups)\\
		\item[2.] Note, $I_d := \Bigg( \begin{matrix} 1 & 0 & \dots & 0\\ 0 & 1 & \dots 0\\ \vdots & \vdots & \ddots & \vdots\\ 0 & 0 & \dots & 1 \end{matrix} \Bigg) \in GL_d(\mathbb{C})$ 
				as $1,0 \in \mathbb{C}$. So, $\forall A \in GL_d(\mathbb{C})$, $I_d \cdot A = A$. So, the identity element exists in $GL_d(\mathbb{C})$\\
		\item[3.] Finally, it is a well known fact that matrix multiplication is associative\footnote{But not necessarily commutative!}. So, $\forall A, B, C \in GL_d(\mathbb{C}), 
			A \cdot (B \cdot C) = (A \cdot B) \cdot C$.
	\end{enumerate}
\end{center}

Essentially, the general linear group is special because it is easy to construct and (as is implied by the name) shares some of the same functions as a group!\footnote{It should be noted that 
the actual structure of the general linear group is that of a field (which affords us more liberties than are present in the otherwise limited structure of a group).} Now that we have two 
structures that are similar to each other, let us see if we can represent one using the other.

Let us now define a function $\mathcal{X}: G \to GL_d(\mathbb{C})$. $\mathcal{X}$ is known as the \emph{character} of $G$ and maps to the appropriate representations in $GL_d(\mathbb{C})$. Now, 
it should be known that the mapping itself is not invariant among all groups. Meaning, that there is no single definition for $\mathcal{X}$ which necessary applies to any two groups $G$ and $H$. 
However, what is absolutely necessary of $\mathbb{X}$ is that it has the following properties:

\begin{defin}
	\begin{center}
		\underline{\textbf{Group Homomorphism}}\\
	\end{center}
	\vspace*{1mm}
	Let $\phi: G \to H$ be defined for groups $(G, \otimes_G)$ and $(H, \otimes_H)$. $\phi$ is a \textbf{group homomorphism} if,
	\begin{enumerate}
		\item[1.] $\forall a,b \in G, \phi(a \otimes_G b) = \phi(a) \otimes_H \phi(b)$.
		\item[2.] For $e_G$ the identity in $G$, $\phi(e_G) = e_H$.
	\end{enumerate}
\end{defin}

\begin{examp}
	\begin{center}
		\underline{\textbf{A Special Representation for $C_4$}}\\
	\end{center}
	Consider the following set and operation, $C_n := \{g^k | \forall k \in [0,n-1]\}$, $\otimes:C_n \times C_n \to C_n,$ $g^i \otimes g^j = g^{i+j \text{ (mod n)}}$. The group 
	$(C_n, \otimes)$ is called the cyclic group of order $n$ and serves as an abstraction of the modulo arithmetic familiar to the integers\footnote{i.e. addition in $\mathbb{Z}_n.$}
	I will construct a character $\mathcal{X}$ that represents the elements of $C_4$ (the cyclic group of order 4) in the complex numbers.\\

	Let $G := \{e, g, g^2, g^3\}$ and $H := \{1, i, -1, -i\}$. Consider, $\mathcal{X}: G \to H$, that maps $g^k \mapsto i^k \in H$. Recall, to show that $\mathcal{X}$ is a representation of $G$, 
	we must show that it is a homomorphism between $G$ and $H$. Note, for $g^m, g^n \in G$, $\mathcal{X}(g^m \otimes g^n) = \mathcal{X}(g^{m+n \text{ (mod 4)}}) = i^{m+n \text{ (mod 4)}}$. Note that 
	$i^{m \text{ (mod 4)}} \cdot i^{n \text{ (mod 4)}} = i^{m+n \text{ (mod 4)}}$. So, $\mathcal{X}(g^m \otimes g^n) = i^m \cdot i^n$. Trivially, $\mathcal{X}(e) = 1$. So, $\mathcal{X}$ is a 
	homomorphism between $G$ and $H$ and is thus a representation!

\end{examp}

The group homomorphism is mechanism that establishes a representation. It is the definition of the group homomorphism that establishes the representation of the elements of $G$ in $H$. In the 
next chapter we will explore how to represent a familiar object from combinatorics in $\mathbb{R}^n$.

\newpage

\section{Partitions and Permutations}



\end{document}
